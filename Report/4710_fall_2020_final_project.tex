\documentclass[conference,compsoc]{IEEEtran}
% \IEEEoverridecommandlockouts

\renewcommand\thesection{\arabic{section}}
\renewcommand\thesubsection{\thesection.\arabic{subsection}}
\renewcommand\thesubsubsection{\thesubsection.\arabic{subsubsection}}

\renewcommand\thesectiondis{\arabic{section}}
\renewcommand\thesubsectiondis{\thesectiondis.\arabic{subsection}}
\renewcommand\thesubsubsectiondis{\thesubsectiondis.\arabic{subsubsection}}

% The preceding line is only needed to identify funding in the first footnote. If that is unneeded, please comment it out.
\usepackage{cite}
\usepackage{amsmath,amssymb,amsthm,amsfonts}
\usepackage{algorithmic}
\usepackage{graphicx}
\usepackage{textcomp}
\usepackage{xcolor}
\usepackage{listings}
\usepackage[ruled]{algorithm2e}
\usepackage[explicit]{titlesec}
\def\BibTeX{{\rm B\kern-.05em{\sc i\kern-.025em b}\kern-.08em
    T\kern-.1667em\lower.7ex\hbox{E}\kern-.125emX}}

% Our custom theorems/properties
\newtheorem{prp}{Property}
\newtheorem{proofprp}{}

\newtheorem{subprp}{Property}[proofprp]
\newtheorem{thm}[subprp]{Theorem}
\newtheorem{defn}[subprp]{Definition}

% \overfullrule=1mm

% -----------------------------------------------------------------
%                   end header
%------------------------------------------------------------------

\begin{document}
\title{An Improved K-Means Clustering Algorithm for One Dimensional Data}

\author{\IEEEauthorblockN{Ryan Froese}
\IEEEauthorblockA{\textit{Dept. of Computer Science} \\
\textit{University of Manitoba} \\
Winnipeg, MB, Canada \\
froeser5@myumanitoba.ca}
\and

\IEEEauthorblockN{James Klassen}
\IEEEauthorblockA{\textit{Dept. of Computer Science} \\
\textit{University of Manitoba} \\
Winnipeg, MB, Canada \\
klass167@myumanitoba.ca}
\and

\IEEEauthorblockN{Tyler Loewen}
\IEEEauthorblockA{\textit{Dept. of Computer Science} \\
\textit{University of Manitoba} \\
Winnipeg, MB, Canada \\
loewent4@myumanitoba.ca}
}

\maketitle

\begin{abstract}
This document is a model and instructions for \LaTeX.
This and the IEEEtran.cls file define the components of your paper [title, text, heads, etc.]. *CRITICAL: Do Not Use Symbols, Special Characters, Footnotes, 
or Math in Paper Title or Abstract.
\end{abstract}

\begin{IEEEkeywords}
Clustering algorithms, Cluster analysis, k-means algorithm, Data analysis
\end{IEEEkeywords}

\section{Introduction}

Clustering is a data mining technique used to arrange large amounts of data into distinct clusters. It is used in many fields including image processing, pattern recognition, bioinformatics, spatial data analysis, and more. The idea of a clustering algorithm is to create clusters containing data points from a dataset where every data point in the same cluster is more similar to one another than to the data points in other clusters (i.e., maximizing the intra-class similarities within a cluster and minimizing inter-class similarities between other clusters). Thus, creating a set of data points in a cluster that are locally more similar to each other than to data points in other clusters. Clustering can help researchers better understand the distribution and structure of large datasets. In Section 2 we introduce the partitioning approach to clustering and describe the k-means clustering algorithm along with some of its shortcomings. We then discuss some commonly used approaches to improving the k-means algorithm. In Section 3 we propose our improved k-means algorithm. We include pseudo code for our algorithm, a formal proof of correctness, and the time complexity of a single iteration. In Section 4 we analyze the experimental results of our algorithm and compare both artificial datasets of different sizes and distributions, and a real-world dataset. In Section 5 we use a real-world dataset containing oceanic and atmospheric data combined with our clustering algorithm to visualize geospatial data. We also discuss any new findings and relationships that become apparent when visualizing the data. Lastly, Section 6 provides a summary of our findings, visualizations of our dataset and potential improvements to our algorithm in the future.

\section{Background}

\subsection{Clustering}

There are three major techniques for clustering: partitioning methods, hierarchical methods, and density-based methods. The partitioning method is used by the original k-means algorithm and our algorithm. The partitioning method takes a dataset of size $n$ and partitions it into some $k$ number of partitions (clusters) where $k \leq n$. Each cluster contains at least one data point, and each data point belongs to exactly one cluster (i.e., distinct clusters). The similarities between items in each cluster are then evaluated using a distance function through an iterative process until all data points are in their current cluster. The centre of a cluster can be calculated using the centroid or medoid methods. The centroid of a cluster is determined by calculating the average of every data point in the cluster while the medoid of a cluster is determined by calculating the median data point of the cluster.

\subsection{k-means Clustering Algorithm}

The k-means algorithm is a widely used clustering algorithm that is easy to understand and implement and can be used across a large variety of problems [1, 2, 3, 4, 5]. We have included the pseudo code of the k-means algorithm as Algorithm 1 for reference. There are two phases in the k-means algorithm. During the initial phase, $k$ random values are chosen from the dataset and assigned as the initial centroids of each of the clusters where $k$ is the number of clusters [1]. In the second phase, the distance from each data point in a cluster to the centroid of every other cluster is calculated. If a data point is closer to a different cluster than the one it currently belongs to, the data point is moved to the closest cluster. After one iteration is complete and all data points have been moved, the centroids for each cluster are recomputed. This process continues until none of the data points move to a different cluster which occurs when the centroids of each cluster do not change. One of the issues with the k-means algorithm is the random selection of initial cluster centroids. This random selection has two main effects on the behaviour of the algorithm. First, it makes the algorithm non-deterministic in the final clusters it produces [1]. This means the number of data points in each cluster can vary greatly depending on the initial centroids chosen. Second, for large datasets, some random selections of initial centroids can greatly reduce the performance of the algorithm as more iterations are required to move data points between clusters [1, 2].

\begin{algorithm}\label{alg-1}
    \caption{k-means clustering algorithm}
    Input:
        \begin{itemize}
            \item $D = \{d_1, d_2, \dots, d_n\}$: Set of $n$ data points in any order
            \item $k$: Number of clusters
        \end{itemize}
    Output:
        \begin{itemize}
            \item A set of $k$ clusters
        \end{itemize}
    Steps:
        \begin{enumerate}
            \item Randomly select $k$ data points from $D$ as initial centroids
        \end{enumerate}
        Repeat until values of centroids do not change:
        \begin{enumerate}
        \setcounter{enumi}{1}
            \item Assign each data point $d_i$ to the cluster with the closest centroid
            \item Calculate the new centroid of each cluster
        \end{enumerate}
\end{algorithm}

\subsection{Related Work}

There have been many improvements to the k-means clustering algorithm [1, 2, 3]. The global k-means clustering algorithm modifies the k-means algorithm by minimizing the distance between data points within the same cluster using the k-means algorithm as a local search procedure [2]. In this algorithm, the local k-means algorithm starts with a single cluster and calculates the centroid of this cluster. The global k-means algorithm then applies the k-means algorithm sequentially on an increasingly large number of clusters to solve a local problem [2]. At the start of each local search, the cluster centroids are initially placed at the currently most optimal position, which is determined by the centroids calculated in the previous iteration’s local search [2]. This improves performance by reducing the number of data points that have to be moved between clusters. The global k-means algorithm produces deterministic results because it starts with a single cluster and computed centroid. It also has improved performance over the k-means algorithm because it always passes the most recently calculated centroid(s) of the current iteration to the next iteration, reducing the number of data points that are moved between clusters [2].

The improved k-means algorithm proposed by K. Nazeer and M. P. Sebastian [1] modifies the original k-means algorithm using a technique to systematically determine the initial centroids of each cluster with better accuracy than when selecting them randomly [1]. The clustering process is then applied to the initial clusters using a heuristic approach for improved efficiency [1]. In this heuristic approach, the distances from each data point in a cluster to their nearest cluster are saved and passed to the next iteration. In the next iteration the distances from the previous iteration are compared to the new distances from each data point in the cluster to the cluster they are closest to. If the distance calculated in the current iteration is less than or equal to the distance calculated in the previous iteration then the data point stays in its current cluster. Otherwise, the distance from the data point to every other cluster is calculated to find the new closest cluster to the data point.

\section{Modified Approach}

\subsection{f-Clustering algorithm}

In this paper, we propose the f-cluster algorithm, a modified version of the original k-means algorithm optimized for single-dimensional data. The f-cluster algorithm has two phases. In the first phase, we use Algorithm 2 to determine the initial cluster centroids which are used in Algorithm 3. To get the initial cluster centroids, Algorithm 2 sorts the data points in ascending order in Step 1 and assigns a centroid value to each $k$ clusters by choosing $k$ data points evenly distributed across the dataset in Step 2. In Step 3, we iterate over each data point exactly once and assign it to the cluster it is closest to. The initial centroids are recalculated in Step 4 to account for data points that have changed clusters in Step 3. Lastly, Step 5 returns the list of initial centroids, a list of the sum of data point values in each cluster, and the number of data points in each cluster.

\begin{algorithm}\label{alg-2}
    \caption{Finding initial centroids}
    Input:
        \begin{itemize}
            \item $D = \{d_1, d_2, \dots, d_n\}$: Set of $n$ data points in any order
            \item $k$: Number of clusters
        \end{itemize}
    Output:
        \begin{itemize}
            \item $C = \{c_1, c_2, \dots, c_k\}$: Set of initial cluster centroids in ascending order
            \item $S = \{s_1, s_2, \dots, s_k\}$: The sum of data point values in each cluster
            \item $N = \{n_1, n_2, \dots, n_k\}$: The number of data points in each cluster
        \end{itemize}
    Steps:
        \begin{enumerate}
            \item Sort $D$ in ascending order
            \item Define $C$ as a linear interpolation with $k$ data points beginning with $d_1$ and ending with $d_k$
            \item For each data point $d_i$:
            \begin{itemize}
                \item Determine the cluster centroid closest to $d_i$
                \item Add the value of $d_i$ to the appropriate element in $S$
                \item Increment the count of the appropriate element in $N$
            \end{itemize}
            \item Re-calculate each centroid $c_i \in C$ as $\frac{s_i}{n_i}$
        \end{enumerate}
\end{algorithm}


In the second phase, we use Algorithm 3 which takes the output of Algorithm 2 and determines the final clusters. Step 1 calculates the positions of the cluster borders by using the number of data points in each cluster as seen in Eq.(1). Once the cluster borders are calculated we calculate the initial average of the cluster border in Step 2 using the centroids of the two clusters neighbouring the cluster border. Step 3 is an iterative looping process that moves the cluster borders one position at a time until they are in the correct position. Step 3 is executed on each cluster border sequentially, only moving the next cluster border when the current cluster border is in the correct position. There are three cases to consider in Step 3. In case 3a, the cluster border continues to move one position to the left as long as the data point to the left of the cluster border is greater or equal to the cluster border average. In case 3b the cluster border continues to move one position to the right as long as the data point to the right of the cluster border is less than the cluster border average. In both case 3a and case 3b, the cluster border will no longer be moved when case 3c is true. In case 3c the cluster border is in the correct position because the data point directly to the left of the cluster border is less than the cluster border average and the data point immediately to the right of the cluster border is greater or equal to the cluster border average. When this is the case, the algorithm continues to the next cluster border. When case 3c is true for every cluster border, then all data points are in the correct clusters; this is the terminating condition of Algorithm 3. When Step 3 has terminated, any clusters containing no data points are removed in Step 4. Then in Step 5, the centroid of each cluster is updated using the cluster sums and cluster counts lists. Since each cluster sum and count is updating continuously in Step 3, calculating the centroids in Step 5 is trivial and does not require iterating over every data point. Finally, Step 6 outputs the list of centroids, sums of each cluster, and the number of items in each cluster.


\begin{algorithm}\label{alg-3}
    \caption{f-cluster algorithm}
    Input:
        \begin{itemize}
            \item $D = \{d_1, d_1, \dots, d_n\}$: Set of $n$ data points in ascending order
            \item $k$: Number of clusters
            \item $C = \{c_1, c_2, \dots, c_k\}$: Initial set of cluster centroids in ascending order
            \item $S = \{s_1, s_2, \dots, s_k\}$: The sum of data points values in each cluster
            \item $N = \{n_1, n_2, \dots, n_k\}$: The number of data points in each cluster
        \end{itemize}
    Output:
        \begin{itemize}
            \item $C = \{c_1, c_2, \dots, c_k\}$: Updated centroids for each cluster
            \item $S = \{s_1, s_2, \dots, s_k\}$: Updated sums of data points values in each cluster
            \item $N = \{n_1, n_2, \dots, n_k\}$: Updated numbers of data points in each cluster
        \end{itemize}
    Steps:
        \begin{enumerate}
            \item Determine the indexes that divide each cluster using $N$.
            \item Compute the cluster border averages by taking the average of the cluster centroids adjacent to each cluster border.
            \item Repeat for each cluster border $b_i$:
            \begin{enumerate}
                \item If the data point immediately before $b_i$ is greater than or equal to the corresponding adjacent cluster border average, move $b_i$ to the left until this is no longer the case. Update the appropriate elements in $S$ and $C$ for each cluster.
                \item If the data point immediately after $b_i$ is less than the corresponding adjacent cluster border average, move $b_i$ to the right until this is no longer the case. Update the appropriate elements in $S$ and $C$ for each cluster.
                \item If the data point immediately before $b_i$ is less than the corresponding adjacent cluster border average, and the data point immediately after the cluster border is greater than or equal to the corresponding adjacent cluster border average, $b_i$ is in the correct position.
            \end{enumerate}
            \item Delete any empty clusters.
            \item Update each $c_i \in C$ as $\frac{s_i}{n_i}$.
        \end{enumerate}
\end{algorithm}


Algorithm 4 is the complete algorithm that combines Algorithm 2 and Algorithm 3. As input, Algorithm 4 takes an unsorted dataset and some number of clusters $k$ and outputs a list of cluster centroids. In phase 1, we use Algorithm 2 to determine the initial cluster centroids as well as the sum and number of data points in each cluster. These outputs are then used in phase 2 where Algorithm 3 performs the full clustering which produces a list of cluster centroids.

% -----------------------------------
% START HERE
% -----------------------------------

\begin{algorithm}\label{alg-4}
    \caption{Full clustering algorithm}
    Input:
        \begin{itemize}
            \item $D = \{d_1, d_1, \dots, d_n\}$: Set of $n$ data points in any order
            \item $k$: Number of clusters
            \item $C = \{c_1, c_2, \dots, c_k\}$: List of cluster centroids
        \end{itemize}
    Output:
        \begin{itemize}
            \item $C = \{c_1, c_2, \dots, c_k\}$: List of cluster centroids
        \end{itemize}
    Steps:
        \begin{enumerate}
            \item Use Algorithm 2 to determine the initial cluster centroids, cluster sums, and cluster data point counts
            \item Use output from Algorithm 2 as input to Algorithm 3 to determine the final cluster centroids
        \end{enumerate}
\end{algorithm}


\subsection{Proof of Correctness}

This algorithm needs a couple requirements in order to function properly:
\begin{enumerate}
    \item There must be no duplicate cluster centroids
    \item Every cluster must contain at least one item in it
\end{enumerate}

Note: if a data point is the same distance from two cluster centroids, this algorithm favours the cluster with the larger centroid.
This case is ignored in this proof, as the chance of this happening is vanishingly small in real world datasets due to floating point precision.

To prove correctness of the algorithm, we must prove the following properties
\begin{prp}\label{closest-centroid}Each item in the dataset will be assigned to the cluster with the closest centroid, and\end{prp}
\begin{prp}\label{border-cross}The cluster borders cannot cross each other (even if the algorithm is done in parallel), as this would cause sum/count calculations to be incorrect.\end{prp}

\subsubsection{Algorithm inputs}
\begin{itemize}
\item $D = \{d_{11}, d_{12}, \dots , d_{1n_1}, \dots, d_{kn_k}\}$ where the dataset $D$ contains elements $d_{ij}$ where $i$ is the cluster and $j$ is the index within the cluster. 
\item $k$, the number of cluster where $k \geq 2$
\item $C = \{c_1, \dots, c_k\}$ where $C$ is a set containing the centroids of each cluster where each $c$ is unique and $c_1 < c_2 < \dots < c_k$. This sequence is constant until the very end of the algorithm.
\item $S = \{s_1, \dots, s_k\}$ where $S$ is a set containing the sums of data points in each cluster
\item $N = \{n_1, \dots, n_k\}$ where $N$ is a set containing the number of data points in each cluster
\item $N_{\text{sum}} = {n_1 + \dots + n_k}$ where $N_{\text{sum}}$ is the total number of data points in $D$
\end{itemize}

\subsubsection{}
In this proof we use the concept of cluster borders separating the data set into its clusters.
This is an integral concept as the algorithm leans heavily on it.
In order to better visualize the intuition of the algorithm, we model the data and associated clusters as a sorted list of data points in clusters separated by cluster borders, i.e. the first border separates clusters 1 and 2, etc.. The set of all cluster borders positions is $B = b_1, \dots, b_{k-1}$. Visually, this looks like the following diagram:

\begin{equation}\label{alg-vis}\begin{aligned}
D=
\underbrace{d_{11}, d_{12}, \dots, d_{1x_1}}_{\substack{s_1 = d_{11} + \dots + d_{1x_1} \\  c_1 = \frac{s_1}{x_1}}}
\overbrace{|}^{\substack{b_1 = \\ x_1}}
% \\
\underbrace{d_{21}, d_{22}, \dots, d_{2x_2}}_{\substack{s_2 = d_{21} + \dots + d_{2x_2} \\  c_2 = \frac{s_2}{x_2}}}
\overbrace{|}^{\substack{b_2 = \\ x_1 + x_2}}
\dots \\
\overbrace{|}^{\substack{b_{k-1} = \\ x_1 + x_2 + \dots + x_{k-1}}}
\underbrace{d_{k1}, d_{k2}, \dots, d_{kx_k}}_{\substack{s_k = d_{k1} + \dots + d_{kx_k} \\  c_k = \frac{s_k}{x_k}}}
\end{aligned}
\end{equation}

% \noindent\fbox{
%     \parbox{225pt}{
%         \subsubsection{Proof outline for Property \ref{closest-centroid}}
%         \textbf{Proof outline for Property \ref{closest-centroid}}
%         \begin{proofprp}\end{proofprp}

%         % \textbf{Property 1.1:}
%         \begin{subprp}\label{1-1}
%             Data points can only be closest to one of the clusters they’re adjacent to or inside of, i.e. $d_{ij}$ is closest to either $c_{i-1}, c_i$ or $c_{i+1}$
%         \end{subprp}

%         % \textbf{Definition 1.2:}
%         \begin{defn}\label{1-2}
%             Definition: a cluster border $b_i$ is in the correct location if all data points before $b_i$ are closer to $c_i$ than $c_{i+1}$, and all data points after $b_i$ are closer to $c_{i+1}$ than $c_i$.
%         \end{defn}

%         % \textbf{Property 1.3:}
%         \begin{subprp}\label{1-3}
%             An equivalent statement: If a cluster border $b_i$ is in the correct location, the item $x$ immediately before $b_i$ satisfies $x < (c_i+c_{i+1})/2$, and the item $y$ immediately after $b_i$ satisfies $y > (c_i+c_{i+1})/2$
%         \end{subprp}
        
%         % \textbf{Property 1.4:}
%         \begin{subprp}\label{1-4}
%             Once the first $i$ cluster borders have been put in the correct location, it follows that all data points before $b_i$ have been assigned to the correct cluster
%         \end{subprp}
        
%         % \textbf{Property 1.5:}
%         \begin{subprp}\label{1-5}
%             Once the last cluster border $b_{k-1}$ has been put in the correct location, all data points after $b_{k-1}$ have been correctly assigned to the last cluster.
%         \end{subprp}

%         \textbf{Conclusion:}
%         By properties 1.4 and 1.5, once the algorithm terminates, all data points have been assigned to the correct cluster.
%     }
% }
% \vspace{15px}

\textbf{Proof of Property \ref{closest-centroid}}

Recall that:
\begin{enumerate}
\item $c_1 < c_2 < \dots < c_k$
\item $d_{11} \leq d12 \leq \dots \leq d_{1n_1} \leq \dots \leq d_{k1} \leq \dots \leq d_{kn_k}$
Due to the properties of means, we know that for any cluster $i$:
$d_{i1} \leq c_i \leq din_i$
\end{enumerate}

\textbf{Property \ref{1-1}}
We are now going to prove that data points can only be closest to one of the clusters they’re adjacent to or inside of,
i.e. $d_{ij}$ is closest to either $c_{i-1}$, $c_i$, or $c_{i+1}$.

Given an item $d_{ij}$ in a cluster $i$ where $j$ is the index of the item inside the cluster, which cluster centroids could $d_{ij}$ conceivably be closest to?
If $i=1$, then it should make sense that $d_{ij}$ is either closest to $c_1$ or $c_2$. It wouldn’t make sense for it to be closer to $c_3$ (or any centroid higher than that), since $c_3 > c_2$.
Putting it more formally: We want to minimize the L2 Norm in 1 dimension $|d_{1i} - c_a|$ over the variable $a$.
By the algebraic definition of absolute values, $|d_{1i}-c3| = c_3-d_{1i}$ because $c_3 > d_{1i}$ and $|d_{1i}-c_2| = c_2-d_{1i}$ because $c_2 \geq d_{1i}$.
Then it follows that:
$|d_{1i} - c_3| > |d_{1i} - c_2| \iff c_3 - d_{1i} > c_2 - d_{1i} \iff c_3 > d_2$ which is one of our given properties.

This same argument applies to any cluster $i$: Data points can only be closest to one of the clusters they’re adjacent to or inside of
i.e. $d_{ij}$ is closest to either $c_{i-1}, c_i$, or $c_{i+1}$

\textbf{Definition \ref{1-2}}
For a cluster border $b_i$ to be in the correct position, all data points to the left of $b_i$ are closer to $ci$ than $c_{i+1}$, and all data points to the right of $bi$ are closer to $c_{i+1}$ than $ci$.

\textbf{Property \ref{1-3}}
For convenience, define $m_i = (c_i+c_{i+1})/2$ as the mean of two adjacent cluster centroids.

\textbf{Lemma:}
if an arbitrary element $d_{ij}$ is less than $m_i$, then it’s closer to $c_i$ than $c_{i+1}$.
Additionally, if an arbitrary element $d_{ij}$ is greater than $m_i$, then it’s closer to $c_{i+1}$ than $c_i$.

\begin{itemize}
    \item If $d_{ij} \leq c_i$ (and hence automatically closer to $c_i$ than $c_{i+1}$):
    \begin{itemize}
        \item Then $d_{ij} < m_i$ since $m_i > c_i$
    \end{itemize}
    \item If $d_{ij} > c_i:$
    \begin{itemize}
        \item Then $|d_{ij} - c_i| = d_{ij} - c_i$ and $|d_{ij}-c_{i+1}| = c_{i+1}-d_{ij}$ (since $d_{ij} < c_{i+1}$ must be true), after which it follows that: $(d_{ij}-c_i) < (c_{i+1}-d_{ij}) \iff 2d_{ij} < c_i+c_{i+1} \iff d_{ij} < (c_i+c_{i+1})/2 \iff d_{ij} < m_i$.
        \item This also applies to $(d_{ij}-c_i) > (c_{i+1}-d_{ij})$: 
        \item $(d_{ij}-c_i) > (c_{i+1}-d_{ij}) \iff 2d_{ij} > c_i+c_{i+1} \iff d_{ij} > (c_i+c_{i+1})/2 \iff d_{ij} > m_i$
    \end{itemize}
\end{itemize}

This extends without loss of generality to an arbitrary element $d_{(i+1)j}$ in cluster $i+1$.
Then by the lemma, our condition for a cluster border being in the correct position is then that
the data point $x$ immediately before the border satisfies $x < m_i$, and the data point $y$ immediately after the border satisfies $y > m_i$,
ensuring that both elements have the shortest distance to the centroid of the cluster it belongs to.
Then by the transitive property, all elements before $x$ are also closer to $c_i$ than $c_{i+1}$, and all elements after $y$ are also closer to $c_{i+1}$ than $c_i$.

\textbf{Property \ref{1-4}}
We will now prove that once the first $i$ cluster borders have been put in the correct location, it follows that all data points before border $b_i$ have been assigned to the correct cluster

Our algorithm starts with the first cluster border at position $b_1$, moving it left/right until it’s in the right place.
Since every element before $b_1$ is closer to $c_1$ than $c_2$, and each item must be closest to either $c_1$ or $c_2$,
we can conclude each item before $b_1$ has been assigned to the correct cluster, i.e. the first cluster.

We then move onto the second cluster border at position $b_2$, and adjust that to the left/right until it’s in the right place.
The same argument as before applies; since every item before $b_2$ is closer to $c_2$ than $c_1$ or $c_3$,
and since every item before $b_2$ must be closest to either $c_1$, $c_2$ or $c_3$, everything before $b_2$ must be assigned to the correct cluster.
We can continue this process for $i$ cluster borders, resulting in every data point before $b_i$ being assigned to the correct cluster.

\textbf{Property \ref{1-5}}
We will now prove that once the last cluster border $b_{k-1}$ has been moved into the correct position, all data points after $b_{k-1}$ have been assigned to the correct cluster.

% TODO: use links to properties
Following from property 1.1 and definition 1.2, all data points after $b_{k-1}$ must be closer to ck than ck-1, and all data points after $b_{k-1}$ must be closer to either ck or ck-1. Therefore, once the last cluster border $b_{k-1}$ has been moved into the correct position, all data points after $b_{k-1}$ have been assigned to the correct cluster.

\textbf{Conclusion:}
% TODO: use links to properties
Therefore, by properties 1.4 and 1.5, once the algorithm terminates, all data points have been assigned to the correct cluster.


% \noindent\fbox{
%     \parbox{225pt}{
%         \textbf{Proof outline for Property \ref{border-cross}}
%         \begin{proofprp}\end{proofprp}

%         % \textbf{Property 2.1:}
%         \begin{subprp}\label{2-1}
%             As a cluster border is being adjusted, it cannot cross an unadjusted border to the left or right.
%         \end{subprp}
        
%         % \textbf{Property 2.2:}
%         \begin{subprp}\label{2-2}
%             As a cluster border is being adjusted, it cannot cross an already-adjusted border to the left or right.
%         \end{subprp}

%         \textbf{Conclusion:}
%         By properties 2.1 and 2.2, cluster borders cannot cross each other, even if the f-cluster algorithm is executed in parallel.
%     }
% }
% \vspace{15px}


\textbf{Proof of Property \ref{border-cross}}
\textbf{Property 2.1:}
Let us start with an un-adjusted cluster border $bi$ surrounded by unadjusted cluster borders $b_{i-1}$ and $b_{i+1}$.
If $bi$ is moved to the left to get to its correct position, it cannot cross to the left of $d_{i1}$ (the data point just to the right of $b_{i-1}$), since $d_{i1} \leq c_i < c_{i+1}$.
This means that $d_{i1}$ is closer to $c_i$ than $c_{i+1}$, but if $b_i$ were to cross to the left of $d_{i1}$, that would imply it was closer to $c_{i+1}$.
This also means that $b_i$ can move at most $n_{i-1}$ data points to the left.

If $b_i$ is instead moved to the right to get to its correct position, it cannot cross to the right of $d_{i+1n_{i+1}}$ (the data point just to the left of $b_{i+1}$).
This is because $d_{i+1n_{i+1}} >= c_{i+1} > c_i$, meaning that $d_{i+1n_{i+1}}$ is closer to $c_i+1$ than $c_i$,
however if $b_i$ crossed to the right of $d_{i+1n_{i+1}}$ that would imply it was closer to $c_i$ than $c_{i+1}$.
This also means that $b_i$ can move at most $n_{i+1}-1$ data points to the right.

Combining the fact that $b_i$ can move at most $n_{i-1}$ spots to the left, or move at most $n_{i+1}-1$ spots to the right, the absolute maximum a cluster border can move in one iteration is $\max(n_{i-1}, n_{i+1}-1)$.

\textbf{Property 2.2:}
Let us start with an unadjusted cluster border $b_i$ surrounded by adjusted cluster borders $b_{i-1}$ and $b_{i+1}$.
If $b_i$ is moved to the left to get to its correct position, it cannot cross to the left of $b_{i-1}$,
as that would imply that an element left of $b_{i-1}$ was closer to $c_{i+1}$ than $c_{i-1}$, when we already know that the element is closest to $c_{i-1}$.

If $b_i$ is instead moved to the right to get to its correct position, it cannot cross to the right of $b_{i+1}$,
as that would imply that an element to the right of $b_{i+1}$ was closer to $c_{i-1}$ than $c_{i+1}$, when we already know that the element is closest to $c_{i+1}$.

\textbf{Conclusion:}
By properties 2.1 and 2.2, cluster borders cannot cross each other, even if the f-cluster algorithm is executed in parallel, and the maximum number of spots a cluster border $b_i$ can be moved is $\max(n_{i-1}, n_{i+1}-1)$

\subsection{Empirical Runtime Analysis}
Since each iteration of the f-cluster algorithm produces identical results to the normal k-means algorithm,
it logically follows that it will take the exact same number of iterations to converge to its final result.
The number of iterations is known to scale with the size of the dataset,
 but is still an area of active research and is highly variable depending on the initial conditions
%  TODO:
 (Note: reference a paper that shows a bound on the number of iterations here),
 and so our analysis of the total runtime will simply include iterations as a variable $i$. 
 \par

We also won’t be giving any mathematical proofs of the run times quoted here as the proof would likely be non-trivial enough to be out of the scope of this project.
\par

For each iteration and each cluster border, we must check if the cluster border needs to be moved, and then move it into the correct position if necessary.
If we say that the total amount the cluster borders moved plus $k-1$ (to account for checking if each cluster border needs to be moved) is $x$,
then the total number of operations done for a single iteration is simply $O(x)$.

For convenience, define $N_\text{sum} = n_1 + \dots + n_k$, so that $N_\text{sum}$ is the number of data points in the dataset

Our first observation is that $x < N_\text{sum}$:
First, note that the maximum number of spots a cluster border $b_i$ can move is $\max(n_{i-1}, n_{i+1}-1)$ as was shown in part 2 of the proof of correctness.
Assume the worst case scenario — that every border except the last moves as far to the left as it can go,
and the last border goes to either the left or right. $b_1$ will move $n_1-1$ spots to the left, $b_2$ will move $n_2-1$ spots to the left, etc... and the last border will move $\max(n_{k-1}-1, n_k-1)$ spots, either to the left or right depending on which is larger.
So we have 

\begin{equation}\begin{aligned}
    x = (n_1-1)+(n_2-1) + \dots + (n_{k-2}-1) + 
    \\ \max(n_{k-1}-1, n_{k-1})+(k-1) \\ 
    = n_1 + n_2 + \dots + n_{k-2} + \max(n_{k-1}-1, n_k-1) + 1 < n_1 + n_2 + \dots + n_{k-2} + \max(n_{k-1}, n_k) < n_1 + n_2 + \dots + n_{k-1} + n_k
\end{aligned}
\end{equation}

thus proving the property that $x < N_\text{sum}$.

As we noted in the proof of correctness, there are at most 3 possibilities for which centroid a given data point is closest to.
This implies a different optimized k-means algorithm in which you only check the nearest 2 or 3 centroids, instead of checking all k centroids as naive k-means does.
The fact that $x < N_\text{sum}$ proves that our algorithm is faster than even this optimized version of k-means.

For the whole algorithm including all iterations, the total number of operations done is simply the sum of x values over all iterations,
which we denote $x_\text{sum}$, making the total number of operations for the clustering portion of k-means $O(x_\text{sum})$.
Below is a plot of how $x_\text{sum}$ scales with the size of the dataset, for a few different distributions of datasets.


% TODO insert figure

As we can clearly see, $x_\text{sum}$ scales linearly with the size of the dataset, although the slope changes based on the distribution.
This is simply the big $O$ constant scaling factor, and we can safely ignore it.
We can then conclude that the actual clustering part of K-means, including all iterations,takes a total of $O(n)$ operations.
Of course, the base requirement to use f-cluster is that the dataset is sorted.
However since real data generally doesn’t come pre-sorted, we can consider the full runtime of clustering a dataset with a size of n using f-cluster to be O(n log n).
This is in stark contrast to other optimized K-means algorithms which take $O(n^2)$ time. (Reference paper with $O(n^2)$ optimized k-means here).

As a side note, if one were to zoom in on the start of the graph posted above, one would notice that the slope isn’t entirely constant.
However this is okay because it quickly converges towards a constant slope as the size of the dataset increases. Below is a graph of $x_\text{sum}/N_\text{sum}$:

In addition, as with normal K-means, our algorithm is parallelizable.
As we noted in the proof of correctness, the cluster dividers cannot cross each other and interfere with sum/count calculations.
This holds true even when done in parallel, and as such parallelizing the algorithm is possible.
If you parallelized both the sorting and the clustering steps (it would be pointless to parallelize just one), 
the total runtime to sort and cluster the data would be $O(\log n + n/(k-1)) = O(n/k)$ (Note: Reference a paper here that mentions the complexity of parallelized sorting algorithms).

\section{Empirical Runtime Analysis}

\section{Visualization}
\subsection{Dataset}

So far we have applied our f-clustering algorithm to various artificial datasets but we now apply it to a real-world dataset containing oceanographic hydrological and marine meteorological data [4]. We then visualize the results in two different formats to make the data easier to interpret. Ocean and atmosphere data requires data preprocessing, conversion, and manipulation to properly understand it due to its vast, complex, and high volume nature [4]. We use the ERA5 global climate and weather dataset which contains hourly data on single levels from 1979 to present. ERA5 uses reanalysis and data assimilation techniques which combine the use of model data and observations to create a complete dataset with pre-interpolated data points aligned with a regular latitude/longitude grid in 0.25 degree intervals. For our visualizations, our dataset contains the sea surface temperature, mean sea level pressure (globally), atmospheric temperature two metres above the surface of the earth, and land-sea mask data at hour 00 for every day in the year 2019. This results in 1,440 data points across the longitudinal range of [0,360] and 720 data points across the latitudinal range of [0,180] for a total of 1,036,800 data points per variable per day. This totals 378,432,000 data points for the entire year per variable or 1,513,728,000 when considering all variables. With four variables, the number of data points in the dataset is incredibly large and therefore difficult to process. We perform the data pre-processing in parallel to greatly reduce the execution time required to process the data.

\subsection{Data Pre-processing}

We perform three data pre-processing algorithms to the original dataset. First, to increase the speed of our f-clustering algorithm without significant loss of data precision, we take the original dataset and significantly reduce its size. This is done by iterating through every data point located at an integer latitude/longitude coordinate and calculating an averaged data point using the grid of data points located four data points above and to the right of the averaged data point. Running this averaging algorithm on each of the variables from the original dataset excluding the sea-land mask variable reduces a variable's size from 1,036,800 to 64,800 data points per day without losing significant amounts of precision. This leads to a total reduction of 1,135,296,000 to 70,956,000 data points. Second, because our two metre atmosphere temperature and mean sea level pressure variables contain data covering the entire globe (including land) we use the land-sea mask variable to remove any data points that were measured above land. This helps keep the focus of our visualization on oceanic data. Last, we remove any averaged data points that contain one or more masked values in their averaging calculation. We parallelize these data pre-processing algorithms which greatly reduces the execution time of the data pre-processing stage.

\subsection{Visualizations}

We created two different types of visualizations to represent the clustered data. The first visualization plots each cluster on an equirectangular projection of the Earth. Each cluster is coloured based on the value of its centroid. The second visualization is a discrete stepped line graph representing the changes in the values of the variable on the y-axis over a single line of latitude of the earth, from the South Pole to the North Pole. Each cluster appears as a step in the line which is helpful for understanding which latitude clusters of similar values are located on the earth. For each variable, we generated a line graph and a map plot for each day of the year. We then created an animated video for each variable and visualization type. This provides an easy way to visually represent any changing trends in the data throughout the year 2019.

Using the visualizations we created, we can gather useful information from the original dataset. When looking at the map plots of sea surface temperature and two metre atmosphere temperature we can see that the regions close to the equator have a higher temperature than the regions closer to the poles, as expected. Furthermore, we can see the distribution of similar data points by observing the size of each cluster. We see in Fig 2 that a large portion of the earth around the equator has a similar air temperature of around 25 degrees Celsius. This is in contrast to the air temperature in the North Pole region that is sub 20 degrees Celsius that covers much less of the map. We can see from observing Fig 1 that the sea surface temperature similarly has cold regions at both poles while the two metre air temperature in Fig 2 is only that cold at the North Pole.

Another key observation we can make is how the clusters change over time throughout the year. When comparing Fig 2 and Fig 4 we can see that on March 20th, 2019 in Fig 2 the regions near the North Pole are much colder than the same region in Fig 4 on September 22nd, 2019. Similarly, in Fig 2 the temperatures directly above Antarctica are warmer than the same region in Fig 4. When considering the cluster with the hottest temperatures, we see that it’s position on the map in Fig 2 moves from just below the equator to above the equator in Fig 4. A similar effect can also be seen in Fig 1 and Fig 3 where the warmest water is located roughly just below the equator in Fig 1 and slightly above the equator in Fig 3.

Using our line graph visualization we can see the clustering results in a completely different format. We observe similar trends as seen in the map plots. When comparing the sea surface temperature in Fig 5 to Fig 7 we see that on September 22nd in Fig 7 the cluster with the coldest temperatures around 75 degrees south is larger than the cluster with the coldest temperatures on March 20th in Fig 5. The opposite is the case for the cluster with the coldest temperatures around 85 degrees north where Fig 6 has more cold data points than the respective cluster in Fig 7. We also observe that the size of the cluster with the warmest temperatures increases significantly in both the sea surface temperature from Fig 5 to Fig 7 and two metre atmosphere temperature from Fig 6 to Fig 8. Lastly, if we consider the overall position of the curve between Fig 5 and Fib 7 and the curve between Fig 6 and Fig 8, it is apparent that the entire curve is shifted farther south of the equator in Fig 5 and Fig 6 and farther north above the equator in Fig 7 and Fig 8. When observing the two metre atmosphere temperature in Fig 6 and Fig 8 we notice significant differences in the temperatures at the south and north poles. In Fig 6 we see that the coldest atmospheric temperature in the South Pole region is around -13 degrees Celsius compared to the -20 degrees Celsius in Fig 8. In Fig 6 we see that the coldest atmospheric temperature in the North Pole is around -30 degrees Celsius compared to the 0 degrees Celsius in Fig 8.

Our observations on the difference in temperature at the polar regions at different times of the year, the sizes of the clusters, and the general shape of the line graphs can ultimately be explained by the movement of the sun. These shifts in atmospheric and sea surface temperature from lower latitudes to higher latitudes and vice versa are caused by the Sun. The Earth receives the highest amount of solar radiation at the point on the Earth's surface where the sun is directly overhead. This point moves north and south throughout the year, moving between the Tropics of Cancer at 23.4 degrees north and the Tropics of Capricorn 23.5 degrees south of the equator respectively [8]. The sun is directly overhead of the Tropics of Cancer during the vernal equinox on March 20th and directly overhead of the Tropics of Capricorn during the autumnal equinox on September 22nd. This explains the shift of sea surface and two metre atmosphere temperature between lower and higher latitudes throughout the year. When the sun is directly overhead the Tropics of Capricorn in the Southern Hemisphere on March 20th we see the largest shift of overall temperatures south of the equator. Similarly, when the sun is directly overhead of the Tropics of Cancer in the Northern Hemisphere on September 22nd we see the largest shift of overall temperatures north of the equator.

As observed from the changes in cluster locations in both sea surface temperature and two metre atmospheric temperature map plots, we notice that the location and sizes of the clusters shift more dramatically in the two metre atmosphere temperature plots than the sea surface temperature plots. This is due to the ocean being a large body of water that does a better job of absorbing and retaining temperature compared to less dense air. This causes the air temperature to increase and decrease at a faster rate from solar radiation when compared to the ocean water temperature.

We have created two videos that show these changes dynamically. They can be found [cite to link].

\section{Conclusion}

In this paper, we proposed the f-clustering algorithm, an improved k-means clustering algorithm to cluster data in a single dimension. The proposed algorithm provided significant performance improvements over the well-known k-means algorithm as well as deterministic results. We then proved the correctness of our algorithm by showing it yields the same results as the traditional k-means algorithm. Using artificial datasets with randomly generated data we empirically analyzed the runtime of the algorithm, observing a runtime of O(n log n) in the worst case; an improvement over other improved k-means algorithms with a time complexity of O(n^2) [1]. We then applied the proposed algorithm to a real-world dataset with oceanic and atmospheric data. To better understand the results of our algorithm and the data it produced we visualized the clusters on a map of the Earth as well as a discrete stepped line graph. We were able to observe trends in the size and locations of clusters on the Earth depending on the time of year.

A clear limitation of the proposed algorithm is that it can only be applied to data in a single dimension. Many datasets that benefit from clustering have a high-dimensionality. As such, a future improvement would be to extend the concepts used by the proposed algorithm to an algorithm that can be applied to high-dimensional data. 

Additionally, improvements can be made to the visualizations presented in this paper. First, the vertical line that appears at the prime meridian (0º longitude) could be removed. This gap in clusters was present even in early visualizations with unmodified data from the dataset. This could be fixed by determining why the data is not available, or interpolate data along the prime meridian to fill in the missing data. Second, the data point averaging algorithm that compresses the dataset could be improved to be more precise when considering masked values in the average calculations. This would result in clusters being plotted closer to land leaving a smaller gap between the edge of clusters and land.

There is a lot of future work related to the topics discussed in this paper that can be done. With the speed of the proposed algorithm combined with our visualization techniques, the visualizations could be expanded to be performed interactively in real-time. This would allow researchers to easily choose between the variable and number of clusters they want to visualize. Another area to expand on is the type of information that is visualized. The relationships between two or more dataset variables could be visualized by creating a function that takes multiple variables and computes a one-dimensional result to be visualized. This could help discover relationships between different components of the global environment such as the relationship between water temperature, air pressure, and air temperature. Another area for future work is parallelization. Because the proposed algorithm has the property that each cluster border will never cross each other it is possible to parallelize its execution by dividing the work into discrete chunks for separate processes to compute without any dependencies.


\end{document}
